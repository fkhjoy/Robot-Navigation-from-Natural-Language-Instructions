\chapter{Related Works}
Navigation needs the agent to reason regarding its relative position to things and the way these relations modification because it moves through the setting. whereas in each learning needs counting on indirect oversight to accumulate spatial information and language grounding, for navigation, the coaching knowledge includes incontestable actions, and for spatial description resolution, annotated target locations. we've studied the matter of reasoning regarding linguistic communication directions to navigate and located some works associated with our project.

\section{Talk The Walk}
In~\cite{DBLP:journals/corr/abs-1807-03367}, they introduce the Talk the Walk, wherever the aim is for 2 agents, a “guide” and a “tourist”, to communicate with one another via natural language so as to attain a standard goal: having the tourist navigate towards the right location. The guide has access to a map and is aware of the target location, however doesn't have idea wherever the tourist is; the tourist features a 360-degree read of the environment, however is aware of neither the target location on the map nor the path to it. The agents got to work along through interaction so as to solve the task successfully. Associate degree example of the task is given in Figure~\ref{fig:3}. 

\begin{figure}[htbp]
    \centering
    \includegraphics[width=1.1\textwidth]{Tourist}
    \caption{ Example of the Talk The Walk task~\cite{DBLP:journals/corr/abs-1807-03367}}
    \label{fig:3}
\end{figure}
In Figure~\ref{fig:3}: in order to help the tourist navigate towards the exact location a “guide” is interacting with the “tourist”  via natural language. The target location is known to the guide but not the location of the tourist. The guide also has access to the map but the tourist doesn't have. Finally the tourist navigate through a 360-degree view of the street environment.
\subsection*{Example:}
Guide: Hello, what are surrounding you?\\ 
Tourist: ACTION:TURNLEFT ACTION:TURNLEFT ACTION:TURNLEFT\\ 
Tourist: Hello, in front of me is a Brooks Brothers \\
Tourist: ACTION:TURNLEFT ACTION:FORWARD ACTION:TURNLEFT ACTION:TURNLEFT 
Guide: Is that a coffee shop or hotel? \\
Tourist: ACTION:TURNLEFT \\
Tourist: It is a stationary shop.\\ 
Tourist: ACTION:TURNLEFT \\
Guide: You need to move to the junction in the northwest corner of the map\\
Tourist: ACTION:TURNLEFT \\ 

Talk The Walk is the first task to combine all three aspects together: in order to observe the environment the tourist first do perception, a dialogue system for achieving common goal through interaction via natural language, action for the tourist to navigate through the environment.

As the main focus of their task is on interactive dialogue, they limit the difficulty of the control problem by having the tourist navigating a 2D grid via discrete actions (turning left, turning right and moving forward)~\cite{DBLP:journals/corr/abs-1807-03367}.
 
 