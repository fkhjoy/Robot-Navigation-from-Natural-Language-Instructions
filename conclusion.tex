\chapter{Conclusion}
As our collective human knowledge getting richer and the purpose of these knowledge is to make human life easy so it is a key component that a robot or an agent that works on behalf of human should have the ability to understand human level language so that humans don't have to take the burden of making sure the instructions s/he gives that is properly translated by the robot. \\

In this work, we introduced robot navigation task using natural language. Typically to navigate a robot we have to use the navigation command that the robot can understand like turn left, turn right etc. But in this work we proposed a method where in an known environment we'll just talk to robot for navigating from one point to another point and we don't have to think about will the robot understand our instruction or not. \\

If one human gives some navigation instruction to another human, the words or language are used between the navigation focused conversation between two humans should also be used for instructing robot. In other words, a robot should supposed to have the ability to understand human level language for navigation related task. \\


\textbf{Limitations} \\
Work that we already have done certainly have some limitations. Here we're addressing some of those limitations. \\

As we used learning method so that it certainly have limitations. We used 10000 data points but still some kind of natural language instruction are missing that we couldn't assign in our dataset. So in future if we give some instruction that is not in our dataset, the model will face some difficulties to extract proper information. \\

Sequence of locations that we are providing as output is an important issue during navigation. But our model sometimes provide some ambiguous output like which is destination it treat that as starting point and which is starting point it treat that as destination. Here some more works left that we couldn't accomplish here. \\

We used only eight locations. So if someone talk to navigate some unknown location rather than indicating that its an unknown location, the robot treat the new location as it is within those eight locations. Which is certainly a major limitations of our work. \\

We only simulate a simple hardware based robot which don't have any access to GPS. But our project shows output in actual world map. So its also a limitations that we didn't prepare proper hardware prototype for the proposed work. \\

\textbf{Future Scope} \\
Based on the limitations, we could easily detect what are some future scope left for our project. \\

Firstly as we didn't develop any speech recognition system here so adding or developing a specialized speech recognition system is one of major section of future work.

If more variation of instruction with complex grammar structure can be added to the dataset then certainly the model will perform better than our current work. \\

To properly detect the sequence of the model it possibly could use a regression along with classification. Classification is used to detect whether a certain location is mentioned in the instruction or not. A regression can be used to predict that which is source and which is destination. So adding a regression functionality is surely a big issue associate with future scope. \\

The hardware we demonstrated here didn't had any GPS access. So making a hardware with GPS access that can be navigated according to the actual map would certainly be a great work that still remain incomplete. \\

We actually proposed a procedure of working with geographical navigation data and natural language instructions. This module can be used in several application like food delivery robot or smart wheelchair etc. 
	
